We continue our journey to the world of divergent series from the one very widely-known formula, it's called a Riemann
zeta function $\zeta(s)$.
For every $s \in \mathbb{R}$ Riemann zeta function is defined as
\begin{equation*}
    \zeta(s) = \sum_{n=1}^{\infty} \frac{1}{n^s}.
\end{equation*}
The following identity holds in terms of zeta function
\begin{equation*}
    \zeta(1 - N) = -\frac{B_N}{N}
\end{equation*}
Consider the case $N=1$
\begin{equation*}
    \sum_{n=1}^{\infty} 1 = \zeta(1 - N)  = - \frac{B_1}{1} = \frac{1}{2}
\end{equation*}
since Bernoulli number $B_1 = \frac{1}{2}$.
Similarly for $N = 2,3$
\begin{align*}
    \sum_{n=1}^{\infty} n &= \zeta(1 - 2) = -\frac{B_2}{2} = -\frac{1}{12} \\
    \sum_{n=1}^{\infty} n^2 &= \zeta(1 - 3) = -\frac{B_3}{3} = 0 \\
    \sum_{n=1}^{\infty} n^3 &= \zeta(1 - 4) = -\frac{B_4}{4} = \frac{1}{120}
\end{align*}

In general,

\begin{equation*}
    \sum_{k=1}^{\infty} k^N = \zeta(1 - N) = -\frac{B_N}{N} = \mathrm{const}
\end{equation*}